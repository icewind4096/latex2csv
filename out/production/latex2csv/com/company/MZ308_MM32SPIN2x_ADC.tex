\def\sectionabstract{模拟/数字转换(ADC) }

\section{模拟/数字转换(ADC)}

\subsec{ADC介绍}

12 位 ADC 是逐次逼近式的模拟-数字转换器(SAR A/D 转换器)。\par

A/D转换器支持多种工作模式:单次转换和连续转换模式,并且可以选择通道自动扫描。A/D转换的启动方式有软件设定、外部引脚触发以及各个定时器启动。\par

窗口比较器(模拟看门狗)允许应用程序检测输入电压是否超出了用户设定的高/低阀值值。\par

ADC的输入时钟不得超过15MHz,它是由 PCLK2 经分频产生。\par 

\subsec{ADC主要特征}

\begin{itemize}


\item{最高12位可编程分辨率的SAR ADC,其中ADC1多达12路外部输入通道和2路内部通道,ADC2多达12路外部输入通道。两个ADC共享8个外部通道,并各有4个独立的外部通道。}


\item{高达1Msps转换速率}
\item{支持多种工作模式:}

\begin{itemize}
	\item{单次转换模式:A/D转换在指定通道完成一次转换}
	\item{单周期扫描模式:A/D转换在所有指定通道完成一个周期(从低序号通道到高序号通道)转换}
	\item{连续扫描模式:A/D转换连续执行单周期扫描模式直到软件停止A/D转换}
\end{itemize}

\item{通道采样时间,分辨率可软件配置}
\item{支持DMA传输}
\item{A/D转换开始条件:}
\begin{itemize}
	\item{软件启动}
	\item{外部触发启动}
	\item{Timer匹配}
\end{itemize}
\item{模拟看门狗,转换结果可和指定的值相比较,当转换值和设定值相匹配时,用户可设定是否产生中断请求}
\end{itemize}

\subsec{ADC功能描述}


下图显示了ADC框图
\fullPic{ADC框图}{src/ADC/picture/424697}{fig:424697}

\subsubsection{ADC开关控制}

通过设置ADCFG寄存器的ADEN位可给ADC上电。当第一次设置ADEN位时,它将ADC从断电状态下唤醒。

ADC上电延迟一段时间后,设置ADCR寄存器的ADST位开始进行转换。

通过清除ADST位可以停止转换,设置ADEN位可置于断电模式。

\subsubsection{通道选择}

包含多路外部输入通道、内部温度传感器通道和内部 1.2V 参考电压通道。每个外部输入通道都有独立的使能位,可通过设置 ADCHS 寄存器的对应位来设置。

\subsec{ADC工作模式}

\subsubsection{单次转换模式}

在单次转换模式下,A/D转换相应通道上只执行一次,具体流程如下:

\begin{itemize}
	\item{通过软件、外部触发输入及定时器溢出置位ADCR寄存器的ADST,开始A/D转换。}
	\item{A/D转换完成时,A/D转换的数据值将存储于A/D的数据寄存器ADDATA和ADDRn中。}
	\item{状态寄存器ADSTA的ADIF位置1。若此时控制寄存器ADCR的ADIE位置1,将产生AD转换结束中断请求。}
	\item{A/D转换期间,ADST位保持为1。A/D转换结束时,ADST位自动清0,A/D转换器进入空闲模式。}
\end{itemize}

\annotate{注:在单次转换模式下,如果软件使能多于一个通道,序号最小的通道被转换,其他通道被忽略。}

\fullPic{单次转换模式时序图}{src/ADC/picture/303814}{fig:303814}

\subsubsection{单周期扫描模式}


在单周期扫描模式下,将对始能的通道按顺序(可通过配置寄存器位SCAN\_DIR选择扫描通道方向)进行一次A/D转换,操作步骤如下:\par

软件或外部触发置位ADST开始,方向设置默认从最小序号通道到最大序号通道的A/D转换,也可按照程序设置,从最大序号通道到最小序号通道的A/D转换。。\par 


每路A/D转换完成后,A/D转换数值将有序装载到相应通道的数据寄存器中,ADIF转换结束标志被设置,如果设置了转换结束中断,则在所有通道转换都完成后产生中断请求。\par 

转换结束后,ADST位自动清0,A/D转换器进入空闲状态。\par 

\fullPic{单周期扫描下使能通道转换时序图}{src/ADC/picture/603371}{fig:603371}

\subsubsection{连续扫描模式}


在连续扫描模式下,A/D转换在ADCHS寄存器中的CHENn位被使能的通道上顺序进行(可通过配置寄存器位SCAN\_DIR选择扫描通道方向),操作步骤如下:\par 

软件或外部触发置位ADST开始,外部触发可软件配置触发延时,方向设置默认从最小序号通道到最大序号通道的A/D转换,也可按照程序设置,从最大序号通道到最小序号通道的A/D转换。。


当所有通道的A/D转换完成一遍后,A/D转换数值将有序装载到相应的数据寄存器中,ADIF转换结束标志被设置,如果设置了转换结束中断,则在所有通道转换都完成后产生中断请求。\par  

只要ADST位保持为1,持续进行A/D转换。当ADST位被清0,A/D转换停止,A/D转换器进入空闲状态。当ADST清0,A/D转换将完成当前转换。\par

\fullPic{连续扫描模式使能通道转换时序图}{src/ADC/picture/550679}{fig:550679} 

\subsubsection{DMA请求}

单周期扫描和连续扫描时通道转换的值存储在各自通道的数据寄存(ADDRn)中,最近一次转换的结果也会保存在ADDATA寄存器中。DMA传输时可以选择传输某个特定通道的数据,或者传输所有扫描通道的结果。\par 

\subsec{数据对齐}

ADCR寄存器中的ALIGN位选择转换后数据储存的对齐方式。数据可以左对齐或右对齐,如下图所示。\par 

\fullPic{数据对齐方式}{src/ADC/picture/730272}{fig:730272}

\subsubsection{可编程分辨率}
ADC 转换有效位数可通过 ADC\_CFG 寄存器中的 RSLTCTL[2:0]位更改,以便加快数据转换速率,有效数据位是在 12 位数据高位对齐。\par

\subsubsection{可编程采样时间}
ADC的时钟ADCLK由PCLK2分频得到,分频系数可通过设置ADCFG寄存器的ADCPRE位来确定,即 PCLK2 / (N+1) / 2分频后作为ADC时钟。\par
设置ADC分辨率为n位(n=8,9,10,11,12),每个通道采样周期为m,采样周期数目可以通过ADC\_CFG 寄存器中的SAMCTL位更改。\par 

采样频率采样时间计算如下:

F$_{\rm sample}$ = F$_{\rm ADCLK}$ / (m + n + 1.5)。

例如:

当 ADCCLK = 15MHz,分辨率配置为12bit,采样时间为 1.5 周期

F$_{\rm sample}$ = F$_{\rm ADCLK}$ / 15。\par 

T$_{\rm CONV}$ = 1.5 + 13.5 = 15 周期 = 1\micro\second

\subsec{外部触发转换}

ADC转换可以由外部事件触发(例如定时器捕获,EXTI线)。如果设置了ADCR寄存器的TRGEN位,就可以使用外部事件触发转换。通过设置TRGSEL位可以选择外部触发源。\par 

具体的外部触发源选择情况,可以参考AD控制寄存器相关位的描述。\par 
 

外部触发可设置延时控制,具体参考ADCR[21:19]的TRGSHIFT的描述。\par 


在触发信号产生后,延时N个PCLK2的时钟周期再开始采样。如果是触发扫描模式,则只有第一个通道采样被延时,其他通道是在上一个采样结束后立即开始。\par 

\subsec{温度传感器}

温度传感器可以用来测量器件周围的温度(T$_{\rm A}$)。 \par 

温度传感器在内部和ADC的内部信号源通道相连接,此通道把传感器输出的电压转换成数字值。当传感器不使用时,可设置寄存器相应位来单独关闭。\par 

温度传感器输出电压随温度线性变化,由于生产过程的变化,温度变化曲线的偏移在不同芯片上会有不同。\par 

内部温度传感器更适合于检测温度的变化,而不是测量绝对的温度。如果需要测量精确的温度,应该使用一个外置的温度传感器。\par 

温度数值计算如下:\par 
T($^{\circ}$C) = (V$_{\rm 25}$ - V$_{\rm SENSE}$) / Avg\_Slope + 25\par
V$_{\rm 25}$:温度传感器在25$^{\circ}$C时的输出电压。\par 
Avg\_Slope:温度传感器输出电压和温度的关联参数。\par
V$_{\rm 25}$和Avg\_Slope的典型值请参考数据手册温度传感器章节。\par

V$_{\rm SENSE}$:温度传感器当前的输出电压\par 
V$_{\rm SENSE}$ = Value * V$_{\rm dd}$ / 4096\par

\subsec{内部基准参考电压}\par 

ADC的内部信号源通道连接了一个内部基准参考电压,大小为1.2V,此通道把1.2v的参考电压输出转换为数字值。

内部参考电压有单独的始能位,可通过设置寄存器的相应位开启或关闭。 

\subsec{窗口比较器模式下AD转换结果监控}

比较模式下提供了上限和下限两个比较寄存器。可通过软件设定CMPCH位选择监控通道。\par 

当CPMHDATA ≥ CPMLDATA时,比较结果大于或等于ADCMPR寄存器的CMPHDATA指定值或者小于CMPLDATA指定值,状态寄存器ADSTA的ADWIF位置1。\par

当CPMHDATA < CPMLDATA时,比较结果如果等于CPMHDATA指定值或者处于两个指定值之间,则状态寄存器ADSTA的ADWIF位置1。
如果控制寄存器ADCR的ADWIE置位,将产生中断请求。\par 

\subsec{ADC寄存器描述}

\regOverview{ADC寄存器概览}{
		
		0x00       &ADC\_ADDATA      &A/D数据寄存器                         &0x00000000   &\autoref{subsubsec:ADC_ADDATA} \\ \hline
		0x04       &ADC\_ADCFG       &A/D 配置寄存器                        &0x00000000   &\autoref{subsubsec:ADC_ADCFG} \\ \hline
		0x08       &ADC\_ADCR        &A/D控制寄存器                         &0x00000000   &\autoref{subsubsec:ADC_ADCR} \\ \hline
		0x0C       &ADC\_ADCHS       &A/D通道选择寄存器                       &0x00000000   &\autoref{subsubsec:ADC_ADCHS} \\ \hline
		0x10       &ADC\_ADCMPR      &A/D窗口比较寄存器                       &0x00000000   &\autoref{subsubsec:ADC_ADCMPR} \\ \hline
		0x14       &ADC\_ADSTA       &A/D状态寄存器                         &0x00000000   &\autoref{subsubsec:ADC_ADSTA} \\ \hline

		0x18 $\sim$ 0x54      &ADC\_ADDR0 $\sim$ 15   &A/D数据寄存器                         &0x00000000   &\autoref{subsubsec:ADC_ADDR} \\	\hline
		0x58                 &ADC\_ADSTA\_EXT        &A/D扩展状态寄存器                     &0x00000000   &\autoref{subsubsec:ADC_ADSTA_EXT} \\	
		
		
}


\regSec{A/D数据寄存器(ADC\_ADDATA)}{ADC_ADDATA}

地址偏移:0x00\par 

复位值:0x0000 0000\par

\regPic{src/ADC/reg/ADC_ADDATA}


\regDes{

		
		31 : 22 & Resvered  &   &      & 始终读为0。\\
		21      & VALID     & r & 0 & 有效标志位(只读)(Valid flag)\par
										 1 = DATA[11:0]位数据有效\par 
										 0 = DATA[11:0]位数据无效\par 
										 相应模拟通道转换完成后,将该位置位,读ADDATA寄存器后,该位由硬件清除。 \\ 
		20      & OVERRUN   & r & 0 & 数据覆盖标志位(只读)(Overrun flag)\par
										1 = DATA[11:0]数据被覆盖\par 
										0 = DATA[11:0]数据最近一次转换结果\par 
										新的转换结果装载至寄存器之前,若DATA[11:0]的数据没有被读取,OVERRUN将置1。读ADDATA寄存器后,该位由硬件清除。
										 \\
										 
		19 : 16 & CHANNELSEL& r & 0 & 该4位显示当前数据所对应的通道(Channel selection)\par
										 0000 = 通道0的转换数据\par
										 0001 = 通道1的转换数据\par
										 0010 = 通道2的转换数据\par
										 0011 = 通道3的转换数据\par
										 0100 = 通道4的转换数据\par
										 0101 = 通道5的转换数据\par
										 0110 = 通道6的转换数据\par
										 0111 = 通道7的转换数据\par
										 1000 = 通道8的转换数据\par 
										 1001 = 通道9的转换数据\par
										 1010 = 通道10的转换数据\par
										 1011 = 通道11的转换数据\par
										 1100 = 通道12的转换数据\par										 
										 1110 = 温度传感器的转换数据\par 
										 1111 = 内部参考电压的转换数据\par
										 其他:无效										 
										 \\										 
										 
										 
		15 : 0 & DATA       & r & 0 & 12位A/D转换结果(Transfer data)\par
										 根据设置左对齐或者右对齐。						 
										 \\					  										 
}

	
\regSec{A/D 配置寄存器(ADC\_ADCFG)}{ADC_ADCFG}

地址偏移:0x04\par 

复位值:0x0000 0000\par 


\regPic{src/ADC/reg/ADC_ADCFG_MT308}


\regDes{

											
		31 : 15 & Resvered         &    &       & 始终读为0。\\
		14      & ADCPRE           & rw & 0  &  ADC预分频 (ADC prescaler)\par
												作为ADCPRE[3:0]的最低位,与bit[6:4]合用
										\\        
		13 : 10 & SAMCTL[3:0]     & rw & 0  & 选择通道x的采样时间(Channel x Sample time selection)\par
										这些位用于独立地选择每个通道的采样时间。在采样周期中通道选择位必须保持不变。\par 										
										0000:1.5周期   \quad \quad \quad 0100:41.5周期   \par
										0001:7.5周期   \quad \quad \quad 0101:55.5周期   \par
										0010:13.5周期  \quad \quad \@ \@ 0110:71.5周期   \par
										0011:28.5周期  \quad \quad \@ \@ 0111:239.5周期  \par
										1000:2.5周期   \quad \quad \quad 1001:3.5周期    \par
										1010:4.5周期   \quad \quad \quad 1011:5.5周期    \par
										1100:6.5周期   \quad \quad \quad 其他:保留       
										\\
										
		9 : 7   & RSLTCTL[2:0]    & rw & 0  & 选择ADCx转换数据分辨率(resolution)\par
										000:12位有效  \quad \quad \quad   001:11位有效\par
										010:10位有效  \quad \quad \quad  011:9位有效\par
										100:8位有效										
										\\
										
										
		6 : 4   & ADCPRE           & rw & 0  & ADC预分频(ADC prescaler)\par
										由软件置‘1’或清‘0’来确定ADC时钟频率。\par 
										bit[14]为0时,PCLK2的2*([6:4]+1)分频后作为ADC时钟,只做偶数分频\par 
										bit[14]为1时,PCLK2的(bit[6:4],bit[14])+2分频后作为ADC时钟
										\\										
										
										
		3      & VSEN     		   & rw & 0  & 内部参考电压使能(Voltage Sensor enable)\par 
		                                       1:内部电压传感器使能\par 
		                                       0:内部电压传感器禁用  \\ 
		2      & TSEN     		   & rw & 0  & 温度传感器使能控制位(Temperature sensor enable)\par 
												1 = 温度传感器使能\par 
												0 = 温度传感器禁止 					
												\\
												
		1      & ADWEN     		   & rw & 0  & A/D窗口比较器使能(ADC window comparison enable)\par 
												1 = A/D窗口比较器使能\par 
												0 = A/D窗口比较器禁用 					
												\\
		0      & ADEN     		   & rw & 0  & A/D转换使能(ADC enable)\par 
												1 = 使能\par 
												0 = 禁用 						
												\\			  										 
}

\regSec{A/D控制寄存器(ADC\_ADCR)}{ADC_ADCR}

地址偏移:0x08\par 

复位值:0x0000 0000\par 


\regPic{src/ADC/reg/ADC_ADCR_MT308}


\regDes{


		31 : 22 & Resvered         &    &       & 始终读为0。\\		
		21 : 19 & TRGSHIFT         & rw & 0     & 外部触发延时采样(External trigger shift sample)\par 
												在触发信号产生后,延时N个PCLK2的时钟周期再开始采样。\par 
												如果是触发扫描模式,其他通道是在上一个采样结束后立即开始。\par 
												0:不延时      \quad \quad \quad \quad   1:4个周期\par 
												2:16个周期    \quad \quad \quad   3:32个周期\par 
												4:64个周期    \quad \quad \quad   5:128个周期\par 
												6:256个周期   \quad \quad \quad   7:512个周期
												\\	
		18 :17  & TRGSEL           & rw & 0     & 外部触发源选择(External trigger selection)\par 
												与位6:4合用
												\\	
		16      & SCANDIR         & rw & 0     & ADC扫描通道顺序(ADC scan direction)\par 
												在单周期扫描或者连续扫描方式时,设置扫描通道的顺序\par
												0:ADC通道选择寄存器按从低到高的顺序扫描\par
												1:ADC通道选择寄存器按从高到低的顺序扫描
												\\													
		
		15 : 12 & CMPCH            & rw & 0  & 窗口比较通道选择(Window comparison channel selection)\par
												0000 = 选择比较通道0转换结果\par
												0001 = 选择比较通道1转换结果\par
												0010 = 选择比较通道2转换结果\par
												0011 = 选择比较通道3转换结果\par
												0100 = 选择比较通道4转换结果\par
												0101 = 选择比较通道5转换结果\par
												0110 = 选择比较通道6转换结果\par
												
												0111 = 选择比较通道7转换结果\par
												1000 = 选择比较通道8转换结果\par 
												1001 = 选择比较通道9转换结果\par 
												1110 = 选择比较温度传感器转换结果\par 
												1111 = 选择比较内部参考电压转换结果\par
		
												其他: 无效
												\\ 
												
												
		11     & ALIGN            & rw & 0  & 数据对齐(Data alignment)\par
												0:右对齐\par 
												1:左对齐 
												\\ 
		10 : 9 & ADMD             & rw & 0  & A/D转换模式(ADC mode)\par
												00:单次转换\par
												01:单周期扫描\par
												10:连续扫描\par
												当改变转换模式时,软件要先禁用ADST位。
												\\
		8      & ADST   		  & rw  & 0      & A/D转换开始(ADC start)\par
												1 = 转换开始\par
												0 = 转换结束或进入空闲状态\par 
												ADST置位有下列两种方式:\par 
												在单次模式或者单周期模式下,转换完成后,ADST将被硬件自动清除。\par 
												在连续扫描模式下,A/D转换将一直进行,直到软件写‘0’到该位或系统复位。
												\\ 

		7      & Reserved           &   &   &       始终读为0。
		                                        \\
	
		6 : 4  & TRGSEL           & rw & 0  & 外部触发源选择(External trigger selection)\par
												ADC1和ADC2的外部触发相同,位[18:17,6:4]选择外部触发源,配置如下:\par
												00000:TIM1\_CC1         \par
												00001:TIM1\_CC2         \par
												00010:TIM1\_CC3         \par
												00011:TIM2\_CC2         \par
												00100:TIM3\_TRGO        \par
												00101:TIM1\_CC4和CC5    \par
												00110:TIM3\_CC1         \par
												00111:EXTI线11         \par
												01000:TIM1\_TRGO        \par
												01001:EXTI线4         \par
												01010:EXTI线5    \par
												01011:TIM2\_CC1         \par
												01100:TIM3\_CC4         \par
												01101:TIM2\_TRGO        \par
												01110:保留         \par
												01111:EXTI线15         \par
												10000:TIM1\_CC4         \par
												10001:TIM1\_CC5         \par
												其他:无效
												\\												
												
		3      & DMAEN     		   & rw & 0  & DMA使能(Direct memory access enable)\par 
												1 = DMA请求使能\par
												0 = DMA禁止 	
												\\
		2      & TRGEN     		   & rw & 0  & 外部硬件触发源(External trigger enable)\par 
												1 = 使用外部触发信号启动A/D转换\par 
												0 = 不用外部触发信号启动A/D转换  
												\\
		1      & ADWIE     		   & rw & 0  & A/D窗口比较器中断使能(ADC window comparator interrupt
				 								enable)\par 
												1 = 使能A/D窗口比较器中断\par 
												0 = 禁用A/D窗口比较器中断 
												\\
		0      & ADIE     		   & rw & 0  & A/D中断使能(ADC interrupt enable)\par 
												1 = 使能A/D中断\par 
												0 = 禁用A/D中断\par 
												如果ADIF置位,A/D转换结束后产生中断请求。 
												\\			  										 
}

\regSec{A/D通道选择寄存器(ADC\_ADCHS)}{ADC_ADCHS}

地址偏移:0x0C\par 

复位值:0x0000 0000\par 


\regPic{src/ADC/reg/ADC_ADCHS_MT308}


\regDes{
		
		31 : 16 & Resvered         &    &       & 始终读为0。\\		
		15       & CHENVS          & rw & 0  & 内部参考电压使能(Voltage Sensor enable)\par
												1 = 使能\par
												0 = 禁用
												\\
        14       & CHENTS            & rw & 0  & 温度传感器使能(Temperature Sensor enable)\par
        										1 = 使能\par
        										0 = 禁用
        \\
			
		13 : 12 & Reserved             &    &    & 始终读为0。\\
		11      & CHEN11           & rw & 0  & 模拟输入通道11使能(Analog input channel 11 enable)\par 
												1 = 使能\par 
												0 = 禁用
        \\
		10      & CHEN10          & rw & 0  & 模拟输入通道10使能(Analog input channel 10 enable)\par 
												1 = 使能\par 
												0 = 禁用
        \\
        9       & CHEN9            & rw & 0  & 模拟输入通道9使能(Analog input channel 9 enable)\par
        										1 = 使能\par
        										0 = 禁用
        \\
		8       & CHEN8            & rw & 0  & 模拟输入通道8使能(Analog input channel 8 enable)\par
        										1 = 使能\par
        										0 = 禁用
        \\

		
		7       & CHEN7            & rw & 0  & 模拟输入通道7使能(Analog input channel 7 enable)\par
												1 = 使能\par
												0 = 禁用
												\\
												
		6       & CHEN6            & rw & 0  & 模拟输入通道6使能(Analog input channel 6 enable)\par
												1 = 使能\par
												0 = 禁用
												\\ 	
		5       & CHEN5            & rw & 0  & 模拟输入通道5使能(Analog input channel 5 enable)\par
												1 = 使能\par
												0 = 禁用
												\\ 	
		4       & CHEN4            & rw & 0  & 模拟输入通道4使能(Analog input channel 4 enable)\par
												1 = 使能\par
												0 = 禁用
												\\ 	
		3       & CHEN3            & rw & 0  & 模拟输入通道3使能(Analog input channel 3 enable)\par
												1 = 使能\par
												0 = 禁用
												\\ 	
		2       & CHEN2            & rw & 0  & 模拟输入通道2使能(Analog input channel 2 enable)\par
												1 = 使能\par
												0 = 禁用
												\\ 	
		1       & CHEN1            & rw & 0  & 模拟输入通道1使能(Analog input channel 1 enable)\par
												1 = 使能\par
												0 = 禁用
												\\ 	
		0       & CHEN0            & rw & 0  & 模拟输入通道0使能(Analog input channel 0 enable)\par
												1 = 使能\par
												0 = 禁用
												\\
												
}


\regSec{A/D窗口比较寄存器(ADC\_ADCMPR)}{ADC_ADCMPR}

地址偏移:0x10\par 

复位值:0x0000 0000\par 

\regPic{src/ADC/reg/ADC_ADCMPR}

\regDes{

		
		31 : 20  & Resvered         &    &       & 始终读为0。\\
		27 : 16 & CMPHDATA         & rw & 0  & 比较数值上限(Compare data high limit)\par
												该12位数值将和指定通道的转换结果相比较。
												\\ 
		15 : 12 & Resvered         &    &       & 始终读为0。\\ 
		11 : 0  & CMPLDATA         & rw & 0  & 比较数值下限(Compare data low limit)\par
												该12位数值将和指定通道的转换结果相比较。
												\\				  										 
}


\regSec{A/D状态寄存器(ADC\_ADSTA)}{ADC_ADSTA}

地址偏移:0x14\par 

复位值:0x0000 0000\par 


\regPic{src/ADC/reg/ADC_ADSTA_MT306}



\regDes{
		
		31 : 20 & OVERRUN          & r & 0   & 通道0 $\sim$ 11的数据覆盖标志位(Overrun flag)\par
												只读。
												\\ 
		19 : 8  & VALID            & r  & 0  & 通道0 $\sim$ 11的有效标志位(Valid flag)\par
		只读。
		\\

		
		7 : 4   & CHANNEL          & r  & 0  & 当前转换通道(Current conversion channel)\par
												该4位在BUSY = 1时表示进行转换中的通道。BUSY = 0时 表示可进行下次转换的通道。
												\\
		3       & Reserved             &    &       &始终读为0。 
												\\
		2       & BUSY             & r  & 0  & 忙/空闲(Busy)\par 
											1 = A/D转换器忙碌\par 
											0 = A/D转换器空闲
											\\
		1       & ADWIF            & rw  & 0  & 比较标志位(ADC window comparator interrupt flag)\par 
												  选择的A/D转换通道,结果大于等于ADCMPHR或小于ADCMPLR,该位置‘1’。\par 
												  该标志位写‘1’清零。
												  \\
		0       & ADIF             & rw  & 0  & A/D转换结束标志位(ADC interrupt flag)\par 
													该位由硬件在通道组转换结束时设置,由软件清除。\par
													1 = A/D转换完成\par 
													0 = A/D转换未完成\par 
													该标志位写‘1’清零。
													\\
}


\regSec{A/D数据寄存器(ADC\_ADDR0 $\sim$ 11,14 $\sim$ 15)}{ADC_ADDR}

地址偏移:0x18\_0x44,0x50\_0x54\par 

复位值:0x0000 0000\par 


\regPic{src/ADC/reg/ADC_ADDR}

\regDes{
		
		31 : 22 & Resvered         &    &       & 始终读为0。\\
		21      & VALID            & r & 0   & 有效标志位(只读)(Valid flag)\par
												1 = DATA[11:0]位数据有效\par 
												0 = DATA[11:0]位数据无效\par 
												相应模拟通道转换完成后,将该位置位,读ADDATA寄存器后,该位由硬件清除。
												\\ 
		20     & OVERRUN           & r & 0      & 数据覆盖标志位(只读)(Overrun flag)\par 
												1 = DATA [11:0]数据被覆盖\par 
												0 = DATA [11:0]数据最近一次转换结果\par 
												新的转换结果装载至寄存器之前,若DATA[11:0]的数据没有被读取,OVERRUN将置‘1’,读ADDATA寄存器后,该位由硬件清除。
												\\ 
		19 : 16  & Reserved        &   &        & 始终读为0。\\
		
		15 : 0   & DATA            & r  & 0  & 通道0 $\sim$ 15的12位A/D转换结果(Transfer data)\par
												根据设置左对齐或者右对齐。\\												
												
}


\regSec{A/D扩展状态寄存器(ADC\_ADSTA\_EXT)}{ADC_ADSTA_EXT}

地址偏移:0x58\par 

复位值:0x0000 0000\par 

\regPic{src/ADC/reg/ADC_ADSTA_EXT}

\regDes{
		31 : 8  & Resvered         &    &     & 始终读为0。\\		
		7 : 6   & OVERRUN          & r  & 0   & 通道的数据覆盖标志位 (Overrun flag)\par
												10: 通道15(V\_SENSOR)\par 
												01: 通道14(T\_SENSOR)
												\\
		5 : 4   & Reserved             &    &     & 始终读为0。
												\\
		3 : 2   & VALID            & r  & 0   & 通道的有效标志位 (Valid flag)\par
												10: 通道15(V\_SENSOR)\par 
												01: 通道14(T\_SENSOR)
												\\
		1 : 0       & Reserved             &    &     & 始终读为0。
												\\												
}